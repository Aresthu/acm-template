\section{C/C++ 语言本身的问题(未完待续)}
	\subsection{源自语言本身的问题}
		
		有时候,我们会因为不熟悉 C/C++ 语言而导致在语言本身上面犯错。下面举一些例子:
		
		\begin{description}
			\item[\%lld 和 \%I64d] 对 long long 来说,应该看 OJ 或比赛的要求。最好直接用 cin、cout。
			\item[\%lf] 对 double 使用 scanf 时才用 \%lf。printf 的时候仍然是 \%f。
			\item[短路运算符] \&\&、||、?: 这三个是短路运算符。举个例子 \lstinline|int b = f() && g();| 如果 f() 返回 0 那么 g() 不会被调用。
			\item[memset] memset 不是初始化数组的函数,而是填充内存的函数。如果不想出错,最好第二个参数只填 0 或 1。例如:
			
\begin{lstlisting}
int p[100];
memset(p,0,sizeof(p));    // p[0] 为 0
memset(p,-1,sizeof(p));   // p[0] 为 -1(0xFFFFFFFF)
memset(p,1,sizeof(p));    // p[0] 为 0x01010101
\end{lstlisting}
	
			\item[getchar()] 先判断它是不是 EOF 然后再转成 char 类型。
			\item[浮点数] 应该都知道不要用“==”和“!=”来判断相等吧。还有一点,取整之前先加个小数,例如:\lstinline|int x = int(pow(2, 10)+0.0001);|。否则容易产生结果为“1023.9999999”然后被截断的问题。
			\item[无符号整数] 注意不要拿负数和无符号整数作比较,例如 \lstinline|if (-1 > strlen("abc"))| 看似不成立,实际上 -1 已经被转换成无符号整数(也许是 4294967295)了……
			\item[左移和右移] 尽量避免对有符号整数使用“$<<$”和“$>>$”。特别地,\emph{不要对负数使用“$>>$”运算符}。
			\item[整数溢出] 不要想当然地认为 INT\_{}MAX 加 1 之后会变成 -INT\_{}MAX。
			\item[string] C++ 标准没有规定 string 内部应该如何存储数据。所以不要对 string 内部结构进行任何假设。
			\item[标准库函数的算法] 有些函数,如 qsort、strstr,C++ 标准没有规定它们应该用什么算法实现。不要假设 strstr 用了 KMP 算法。
		\end{description}
	
	\subsection{未定义行为}
		如果发现在 OJ 上用 g++ 提交不能过而 C++ 能过(或者反过来),那么很有可能是因为你的代码中存在未定义行为。这样的话,就算你 AC 了,你的程序也是有错误的。
		
		在 C/C++ 中,除了标准中规定的行为,其他的都是未定义行为。对于未定义行为,编译器不同,结果可能一样,也可能不一样。以下是一些常见的例子。
		
		\begin{description}
			\item[未初始化的变量] 变量未初始化就使用。如 \lstinline|int k; k++;|
			\item[多个操作数] 如果有两个以上的操作数,或者函数有多个参数,C++ 可没规定到底应该先计算哪个(短路运算符除外):
			
\begin{lstlisting}
int b = f() + g();  
printf("%d %d", f(), g());
// 不能确定到底是先执行 f() 还是 g()
\end{lstlisting}

			\item[访问越界] 超过数组范围的访问。例如 \lstinline|int p[4]; p[4]=1;|
			\item[产生副作用的运算符] 不恰当地使用“++”、“--”或其他产生副作用的运算符也会引发未定义行为。例如谭浩强书上经典的 \lstinline|a=a++;| 和 \lstinline|a+=a-=a*a;|。C++ 标准没有规定它们应该怎样计算,所以它们的计算顺序是无法确定的。
			\item[无返回值的函数] 函数没有返回值。
			\item[函数返回引用局部变量的指针] 因为函数结束,局部变量所占内存空间就会被销毁。例如:

\begin{lstlisting}
int * f()
{
	int p=100;
	return &p;
}
\end{lstlisting}
		\end{description}
