\subsubsection{求全部子集}
	示例:1、12、123、13、2、23、3
	
	代码和求一般组合类似,区别在以下两点:首先,$m$ 和 $n$ 是相等的;其次,不管已经产生了多少位数,只要函数被调用,就立刻输出已经产生的结果。
	
	若输入数据有重复,统计本质不同的数据即可。
	
	\lstinputlisting{ch6/codes/产生全部子集.cpp}