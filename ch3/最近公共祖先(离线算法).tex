\subsection{离线算法 (Tarjan)}
	求 LCA 的 Tarjan 算法是一个经典的离线算法。
	
	Tarjan 算法用到了并查集。LCA 问题可以用 $O(n+Q)$ 的时间来解决,其中 $Q$ 为询问的次数。
	
	Tarjan 算法基于深度优先搜索的框架。对于新搜索到的一个结点,首先创建由这个结点构成的集合,再对当前结点的每一个子树进行搜索,每搜索完一棵子树,则可确定子树内的 LCA 询问都已解决,其他的 LCA 询问的结果必然在这个子树之外。
	
	这时把子树所形成的集合与当前结点的集合合并,并将当前结点设为这个集合的祖先。之后继续搜索下一棵子树,直到当前结点的所有子树搜索完。这时把当前结点也设为“已被检查过的”,同时可以处理有关当前结点的 LCA 询问,如果有一个从当前结点到结点 $v$ 的询问,且 $v$ 已被检查过,那么,由于进行的是深度优先搜索,所以当前结点与 $v$ 的最近公共祖先一定还没有被检查,而这个最近公共祖先的包含 $v$ 的子树一定已经搜索过了,因此这个最近公共祖先一定是 $v$ 所在集合的祖先。
	
	\lstinputlisting{ch3/codes/最近公共祖先(离线tarjan).cpp}