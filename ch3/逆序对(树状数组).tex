% ********************************
% 备注:不保证正确性!
% ********************************
\subsection{树状数组}
	\emph{备注:不保证以下内容是正确的。}

	由于我们只是看两个数之间的大小关系,所以可以对序列中的数进行离散化。即按照大小关系把 $a_1$ 到 $a_n$ 映射到 1 至 $num$ 之间 ($num$ 为不同数字的个数),保证仍然满足原有的大小关系。
	
	这样,本题就转化成了:对于一个数 $a_i$,在它后面有多少个比它小的数?
	
	处理的时候,我们从第 $n$ 个数倒着处理,用树状数组维护一个 $cnt[]$ 数组,其前缀和 Query($x$) 表示 “到当前处理的第 $i$ 个数为止,映射后值为 1 到 $x$ 之间的数字一共有多少个”,换句话说,如果 $x$ 对应的是 $a[i]$,那么比 $a[i]$ 小的数的个数就是 Query($x-1$)。
	
	对于维护,只要在该次查询结束后进行修改即可,即 Change($x$,1)。整个算法的时间复杂度为 $O(n\log n)$。
	
	\lstinputlisting{ch3/codes/树状数组求逆序对.cpp}