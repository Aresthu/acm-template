\subsection{C++ 的 set 和 map}
	map (映射)、multimap (多重映射)、set (集合)、multiset (多重集合) 属于关联容器,头文件分别位于 \header{map} 和 \header{set} 中。

	map 和 set 的区别:set 实际上就是一组元素的集合,但其中所包含的元素的值是唯一的,且是按一定顺序排列的。集合中的每个元素被称作集合中的实例。其内部通过链表的方式来组织;而 map 提供一种 “键—值” 关系的一对一的数据存储能力,类似于字典。其 “键” 在容器中不可重复,且按一定顺序排列。由于其是按链表的方式存储,它也继承了链表的优缺点。

	multiset 和 set 不同之处在于,multiset 中元素的值可以不唯一。multimap 也类似,在 multimap 中 “键” 可以不唯一。
	
	关联容器的特点:
	
	\begin{enumerate}
		\item 关联容器对元素的插入和删除操作比 vector 快,但比 list 慢。
		\item 关联容器对元素的检索操作比 vector 慢,但是比 list 要快很多。关联容器查找的复杂度基本是 $O(\log n)$。
		\item set 是内部排序的,这与序列容器有着本质的区别。
	\end{enumerate}
	
	下面是有关 set 和 multiset 的用法:
	
\begin{lstlisting}
// 1. 定义
set < int, less<int> > s1;		// 集合内部升序排列
set < int, greater<int> > s2;	// 集合内部降序排列
// 和其他容器一样,set 也可以用预定义的区间来初始化。

// 2. 查询
s.count(10);	// 返回 s 中值为 10 的具体数目。但对于 set 来说,返回值不是 0 就是 1。
s.empty();		// 判断集合是否为空集。
s.size();		// 返回集合的元素数量。

// 3. 插入和删除
s.insert(e);	// 将e插入到set中。
// 注意,insert 的返回值是一个 pair,其 first 是指向插入后元素的迭代器,
// second 表示插入是否成功(如果其 second 为 false,说明元素已经存在)。
s.insert(begin, end);	// 将区间 [begin, end) 中的值插入到 s 中。
s.erase(e);				// 将 e 删除并返回剩余的 e 的数量。
s.erase(pos);			// 将 pos 处的元素删除。
s.erase(begin, end);	// 将 [begin, end) 处的元素删除。

// 4. 迭代器:begin、end、rbegin、rend 分别返回正向和反向迭代器。
\end{lstlisting}
	
	下面是有关 map 和 multimap 的用法:
	
\begin{lstlisting}	
// 1. 定义
map<int, string> m;		// 键的类型是 int,值的类型是 string。
// 和其他容器一样,map 也可以用预定义的区间来初始化。

// 2. 查询
// m.at(3) 或 m[3]:返回一个引用,指向键为3时的对应值。注意,它不是数组下标!
// 如果元素不存在,map 会自动建立这个元素。
m.count(3);		// 返回 s 中键为 3 的具体数目。但对于 map 来说,返回值不是 0 就是 1。
m.find(3);		// 返回指向键为 3 的元素的迭代器。如果不存在,则返回 m.end()。
m.empty();		// 判断映射是否为空映射。
m.size();		// 返回映射的元素数量。

// 3. 插入和删除
m.insert(begin, end)	// 将区间 [begin, end) 中的值插入到 s 中。该区间应该是 map 类型的。
m.insert(make_pair(10, "Hello"));	// 将元素插入到map中。需要<utility>
m.erase(e);			// 将键为 e 的元素删除。返回值为被删除的 e 的数量。
m.erase(pos);		// 将 pos 处的元素删除。
m.erase(begin, end);	// 将 [begin, end) 处的元素删除。

// 4. 迭代器:begin、end、rbegin、rend 分别返回正向和反向迭代器。
\end{lstlisting}