\subsection{基本二叉树}
	二叉树是递归定义的:它要么为空,要么由根结点、一个左子树和一个右子树构成,且左子树和右子树都是二叉树。

    \subsubsection{实现}
	    二叉树可以用数组保存,也可以用链表保存。根据问题的实际需求,链表既可以只保存左右子树结点,也可以把父节点包括进去。
	    
	    \lstinputlisting{ch3/codes/基本二叉树.cpp}
	    
	    对于创建结点函数,有人希望能够把树根直接扔到函数参数里,而不是写成赋值的形式。这时可以这样定义函数:
	    
	    \begin{lstlisting}
	    void createNode(node* & p)
	    \end{lstlisting}
	    
	    引用是 C++ 的概念。上面代码表示“代表一个指针的引用”。
    
    \subsubsection{完全二叉树}
	    一棵深度为 $k$,且有 $2^k-1$ 个节点的二叉树,称为满二叉树。在一棵二叉树中,除最后一层外,若其余层都是满的,并且最后一层要么是满的,要么是在右边缺少连续若干节点,则此二叉树为完全二叉树。
    
		由于完全二叉树排列密集,所以可以用一个一维数组来保存二叉树而不造成太大浪费。
		
		假设一个完全二叉树的结点数为 $n$,树根的序号是 0,那么可根据表 \ref{tab:ch3_jbecs_wqecs} 来定位。
		
		\begin{table}[htb]
			\centering
			\begin{tabular}{ccc}
				\toprule
				关系 &	函数 &	备注	\\
				\midrule
				$r$ 的父亲 &	$(r-1)/2$ &	$r\neq 0$	\\
				$r$ 的左儿子 &	$r*2+1$ &	$2r+1<n$	\\
				$r$ 的右儿子 &	$r*2+2$ &	$2r+2<n$	\\
				$r$ 的左兄弟 &	$r-1$ & $r$ 为偶数且 $0<r<n$	\\
				$r$ 的右兄弟 &	$r+1$ & $r$ 为奇数且 $r+1<n$	\\
				$r$ 是否为叶子? &	$r\geqslant n/2$ &	$r<n$	\\
				\bottomrule
			\end{tabular}
			\label{tab:ch3_jbecs_wqecs}
			\caption{完全二叉树各结点关系}
		\end{table}
		
		
		