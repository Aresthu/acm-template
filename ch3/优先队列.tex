\subsection{STL 中的优先队列}
	头文件:\header{queue}

	priority\_{}queue 基于 vector 实现\footnote{模板接受三个参数,第二个就是容器类型。一般使用 vector,也可使用 deque。}。普通的队列是先进先出,而优先队列是按照优先级出队,即无论入队顺序如何,出队的都是最大 (最小) 值。
	
	priority\_{}queue 位于 \header{queue},而 greater 和 less 存在于 \header{functional}。
	
	可以用以下几种方式定义优先队列 (假设 arr 是一个有 10 个元素的数组):
	
\begin{lstlisting}
// 元素为 int 类型,最大值先出列。
priority_queue<int> q1;
// 元素为 int 类型,最小值先出列。注意两个 > 之间有空格。
priority_queue< int,vector<int>,greater<int> > q2;
// 元素为 float 类型,最大值先出列,用现有数组初始化。
priority_queue<float> q3(arr, arr+10);
\end{lstlisting}

	优先队列支持的操作有:push、top (不是 front)、pop、empty 和 size。
	
	如果需要使用自己的结构体,你需要重载复制构造函数和 “$>$” 或 “$<$” 运算符。less 对应 “$<$”,表示最大值先出列;greater 对应 “$>$”,表示最小值先出列。

\begin{lstlisting}
struct MyStruct
{
	int v;
	MyStruct(int i):v(i) {}
	
	bool operator < (const MyStruct & b) const {return v < b.v;}
};
priority_queue < MyStruct,vector<MyStruct>,less<MyStruct> > q;
\end{lstlisting}
