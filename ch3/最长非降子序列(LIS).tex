\subsection{最长非降子序列 (LIS)}
	\paragraph{问题} 有一个序列 $a_1,a_2,a_3,\cdots,a_n$ 共 $N$ 个元素。现在要求从序列找到一个\emph{长度最长}、且\emph{前面一项不大于它后面任何一项}的子序列,元素之间\emph{不必相邻}。$N\leqslant 100000$。

	\paragraph{思路} 如果用动态规划求解,那么状态转移方程为:$f(i)=\max\{f(j)\}+1 (a_j>a_i \textrm{且} i>j)$。时间复杂度为 $O(n^2)$,很明显,时间不够用。
	
	超时的原因是大部分时间都耗在了寻找 $\max\{f(j)\}$ 上面。

	根据这一点,我们可以对算法进行改进。假设数组 $C[]$ 是 “到目前为止找到的一个非降子序列”,并且使 $C[]$ 是有序的\footnote{尽管不断加入的元素会破坏这个序列,不过最长序列的最后一个数是不会被覆盖的。因此,这样做可以保证 $C[]$ 的长度与最长序列的长度相等。},那么我们就可以利用二分查找来确定 $f[j]$。时间复杂度由 $O(n^{2})$ 降到了 $O(n\log n)$。
	
	\lstinputlisting{ch3/codes/最长非降子序列.cpp}