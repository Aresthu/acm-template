\subsection{基本二叉堆}
\paragraph{二叉堆}
	二叉堆是完全二叉树。按照树根大小,二叉堆可分为最大值堆和最小值堆。
	
	二叉堆的特点:
	
	\begin{enumerate}
		\item 最大 (小) 值堆中,结点一定不小 (大) 于两个儿子的值。
		\item 在堆中,两兄弟的大小没有必然联系。
		\item 最大 (小) 值堆的根结点是整个树中的最大 (小) 值。
	\end{enumerate}
	
\paragraph{实现}
	本节的二叉堆是最大值堆,修改代码中的标记部分可以变成最小值堆。

	由于是完全二叉树,所以可以直接用一维数组保存。数组的下标是从 0 开始的。

	二叉堆的操作有:

	\begin{enumerate}
		\item 插入:在堆中插入元素,首先要把元素放到末尾,然后通过不断往上“拱,把元素“拱”到正确的位置。
		\item 用现有值初始化:最快的方法不是挨个插入,而是直接调整数组元素的顺序,使其符合堆的性质。
		\item 查找:查找最值是最快的------直接访问树根就可以了。不过,用堆查找其他值就很慢了。因此,可以考虑再使用一个适合查找的辅助数据结构,例如二叉排序树。
		\item 删除:把堆中最后一个元素 (就是一维数组存储所对应的最后一个元素) 放到待删除元素的位置,将元素总数减一,然后调整各元素的顺序。
	\end{enumerate}
	
	\lstinputlisting{ch3/codes/普通二叉堆.cpp}