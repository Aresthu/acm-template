\subsubsection{KMP 算法}
\paragraph{KMP 算法的原理}
	假设原串 S 为 abababacabcababacbc,模式串 P 为 ababacb。

	按照朴素算法的做法,发现不匹配之后,应该把模式串向右移动一位。而 KMP 算法会根据\emph{预处理的结果}来决定应该把模式串向右移动多少位。
	
	朴素算法的尝试:
	
	\begin{lstlisting}
S:  abababacababababacbc
P:  ababacb
P1:  ababacb
P2:   ababacb
P3:    ababacb
......
Pn:             ababacb
	\end{lstlisting}
	
	KMP 算法的尝试:
	
	\begin{lstlisting}
S:  abababacababababacbc
P1: ababacb                // P[5] 时失败,向右移动 2 位,下次从 P[3] 开始
P2:   ababacb              // P[6] 时失败,向右移动 6 位,下次从 P[0] 开始
P3:         ababacb        // P[5] 时失败,向右移动 2 位,下次从 P[3] 开始
P4:           ababacb      // P[5] 时失败,向右移动 2 位,下次从 P[3] 开始
P5:             ababacb    // 成功	
	\end{lstlisting}
	
	现在看一下 KMP 是怎样知道模式串向右移动的位数的。为了看得清楚,我们进行一下处理:
	
	\begin{lstlisting}
S:  ababa bac ababa ba bacbc
P1: ababa|cb  
P2:   aba|bac|b
P3:          |ababa|cb
P4:             aba|ba|cb
P5:               a ba|bacb
	\end{lstlisting}
	
	我们在使匹配失败的字母的前面加上了竖线。设 P[0:x] 为“到目前为止,成功匹配部分的长度为 x”。可以看到,在每组竖线的前面,移动后的 P[0:j] (在继续匹配之前,P 只剩 $j$ 个字母) 和移动前的 P[0:i] 的每一个字母仍然是\emph{对应的}。也就是说,\emph{P[0:j] 实际上是 P[0:i] 的前缀}。这样,我们就可以减少很多无效的右移。
	
	所以,我们要做的就是开一个 next 数组。对于 next[j]=k,我们认为\emph{P[0:k] 是 P[0:j] 的最长的前缀}。把 next 数组算好之后,就可以正式开始和原串的匹配了。
	
\paragraph{实现}
	以下是 KMP 算法的代码:
	
	\lstinputlisting{ch3/codes/kmp.cpp}