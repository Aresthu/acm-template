\section{欧拉路径}
	找欧拉路径,直观地讲,就是解决“一笔画”问题。
	
	\subsection{无向图的欧拉路径}
		两个定义:在一个无向图中,一条包含所有边,且其中每一条边只经过一次的路径叫做欧拉通路。若这条路径的起点与终点为同一点,则为欧拉回路。
		
		判定一个图是否存在欧拉通路或欧拉回路的根据如下:
		
		\begin{enumerate}
			\item 一个图有欧拉回路当且仅当它是连通的 (即不包括 0 度的结点) 且每个结点都有偶数度。
			\item 一个图有欧拉通路当且仅当它是连通的且除两个结点外,其他结点都有偶数度。在此条件下,含奇数度的两个结点一定是欧拉通路的起点和终点。
		\end{enumerate}
		
		\lstinputlisting{ch5/codes/欧拉路径.cpp}
		
	\subsection{有向图的欧拉路径}
		对于有向图来说,我们可以把“奇点”和“偶点”的定义稍作修改,认为出度和入度相等的点为“偶点”,出度和入度之差为 1 的点为“奇点”。
		
		很明显,如果某个点的出度和入度之差大于 1,那么显然不可能做到“一笔画”。
		
		当然,上面代码中的“G[i][j]=G[j][i]=false;”一句也得跟着改成单向的“G[i][j]=false;”。