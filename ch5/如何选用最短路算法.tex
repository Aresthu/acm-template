\subsection{如何选用}
	Floyd 是多源最短路算法,而另外几种是单源最短路算法。Floyd 可以一次性计算任意两个点之间的最短路。
	
	下面按照稀疏图和稠密图来进行分类:
	
	\paragraph{稠密图} 邻接矩阵比邻接表合适。
	\begin{itemize}
		\item 如果 n 不大,可以用邻接矩阵的话,Dijkstra 是最好的选择,SPFA 也可以。
		\item n>300 时最好不要用 Floyd 算法。
		\item n>200 时尽量不要用 Bellman-Ford 算法。
	\end{itemize}
	
	\paragraph{稀疏图} 邻接矩阵会造成一定的浪费。可能邻接表更合适。
	
	\begin{itemize}
		\item 超过 3000 个点就不要用朴素的 Dijkstra 算法了。
		\item 边数超过 50000 时尽量不要用 Bellman-Ford 算法。
		\item 超过 500000 条边就不要用 SPFA 加邻接表了。
		\item 点和边数很多的时候,优先队列优化的 Dijkstra 是更合适的。不过,由于优先队列自身的原因,速度会比堆慢一些。
	\end{itemize}
	
	\paragraph{最短路算法的时间复杂度} 见表 \ref{tab:ch5_zdl}
	
	\begin{table}[htb]
		\centering
		\begin{tabular}{lll}
			\toprule
			算法  &  图的表示法  &  时间复杂度	\\
			\midrule
			Dijkstra &	邻接矩阵(邻接表也可) &	$O(n^{2})$	\\
			优先队列优化的 Dijkstra &	邻接表 &	一般 $O[(m+n)\log m]$,密集图 $O(n^{2}\log m)$ \\
			Bellman-Ford &	边目录 &	$O(mn)$	\\
			优化的 Bellman(SPFA)	& 邻接表 &	$O(km)$,其中 k 可认为是常数。	\\
			Floyd &	邻接矩阵 &	$O(n^{3})$	\\
			\bottomrule
		\end{tabular}
		\label{tab:ch5_zdl}
		\caption{最短路算法的时间复杂度比较}
	\end{table}
	
