\section{启发式搜索}
	启发搜索与盲目搜索的根本区别在于,在扩展结点时会估算一下这个结点的~``前途如何'',即~$h(S)$,也就是对~$N$~到目标的接近程度的\emph{估计}。然后对这些~$h$~从小到大排序,然后再扩展结点。
	
	很明显,这样做还不能保证得到最优解。因此我们引入了~A*~算法。在计算~``前途''~的时候,我们采用的是~$f(S)$~而不是~$h(S)$。一般情况下,我们会令~$f(S)=g(S)+h(S)$,其中~$g(S)$~是目前已知的从起点到~$S$~的最优路径的长度\footnote{一般就是结点在搜索树中的深度。}。在这里,$g$~取决于搜索树,而~$h$~是状态的函数,与搜索树无关。
	
	$h$~函数需要满足一些条件:
	
	\begin{itemize}
		\item $h(S)$~函数应该满足~$0 \leqslant h(S) \leqslant h^*(S)$,其中~$h^*(S)$~为~$S$~到目标结点的\emph{真实}最优路径长度。
		\item $h(S)$~函数应该具有单调性。即如果~$h(S)=0$,且对任意结点~$S$~和它的儿子~$S'$,满足~$h(S) \leqslant c(S,S')+h(S')$。其中~$c(S,S')$~表示从~$S$~向~$S'$~转移需要付出的实际代价。
		\item 第二条从另一个角度来讲,就是要求~$h$~不能减少得太快。
	\end{itemize}
	
    \subsection{A×算法}

    \subsection{IDA×算法}

